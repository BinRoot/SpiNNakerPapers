\documentclass[journal]{./sty/IEEEtran}

% *** CITATION PACKAGES ***
\usepackage{cite}
% cite.sty was written by Donald Arseneau
% V1.6 and later of IEEEtran pre-defines the format of the cite.sty package
% \cite{} output to follow that of IEEE. Loading the cite package will
% result in citation numbers being automatically sorted and properly
% "compressed/ranged". e.g., [1], [9], [2], [7], [5], [6] without using
% cite.sty will become [1], [2], [5]--[7], [9] using cite.sty. cite.sty's
% \cite will automatically add leading space, if needed. Use cite.sty's
% noadjust option (cite.sty V3.8 and later) if you want to turn this off.
% cite.sty is already installed on most LaTeX systems. Be sure and use
% version 4.0 (2003-05-27) and later if using hyperref.sty. cite.sty does
% not currently provide for hyperlinked citations.
% The latest version can be obtained at:
% http://www.ctan.org/tex-archive/macros/latex/contrib/cite/
% The documentation is contained in the cite.sty file itself.


% *** MATH PACKAGES ***
%
\usepackage[cmex10]{amsmath}
% A popular package from the American Mathematical Society that provides
% many useful and powerful commands for dealing with mathematics. If using
% it, be sure to load this package with the cmex10 option to ensure that
% only type 1 fonts will utilized at all point sizes. Without this option,
% it is possible that some math symbols, particularly those within
% footnotes, will be rendered in bitmap form which will result in a
% document that can not be IEEE Xplore compliant!
%
% Also, note that the amsmath package sets \interdisplaylinepenalty to 10000
% thus preventing page breaks from occurring within multiline equations. Use:
% \interdisplaylinepenalty=2500
% after loading amsmath to restore such page breaks as IEEEtran.cls normally
% does. amsmath.sty is already installed on most LaTeX systems. The latest
% version and documentation can be obtained at:
% http://www.ctan.org/tex-archive/macros/latex/required/amslatex/math/


\usepackage{url}


% correct bad hyphenation here
\hyphenation{net-works SpiNNaker}


\begin{document}
%
% paper title
% can use linebreaks \\ within to get better formatting as desired
\title{Modifying Synaptic Connections On The\\Spiking Neural Network Architecture (SpiNNaker)\\In Real-Time
}


\author{Nishant Shukla ~
        Matthew Frazier }

% The paper headers
\markboth{Journal of Neuroscience Methods,~Vol.~x, No.~y, January~2014}%
{Shell \MakeLowercase{\textit{et al.}}: Synaptogenesis on the SpiNNaker}
% The only time the second header will appear is for the odd numbered pages
% after the title page when using the twoside option.


% make the title area
\maketitle

\begin{abstract}
Artificial Neural Networks is a promising approach to study human brain computation. Recent computer architecture design of a low-power 72-core processor by the University of Manchester (SpiNNaker) has made it easier to study highly parallel networks. We designed an algorithm on the SpiNNaker chip to enable synaptic removal, addition, and randomization on a neural network topology during run-time. Additionally, we explored scalability issues and unintended pitfalls with this approach.

\end{abstract}

\begin{IEEEkeywords}
Neural Networks, Synaptogenesis, SpiNNaker
\end{IEEEkeywords}


\section{Introduction}
\IEEEPARstart{T}{he} human brain is fast and low in energy. [TODO: Matt] blah blah blah. One property of the brain is it's plasticity (cite Lashley, or some Psyc studies). Our approach to neural networks is greatly influenced by this biological idea, and we want to enable networks to self-modify. 
% You must have at least 2 lines in the paragraph above
With energy efficiency in mind, we enable this functionality on the Spiking Neural Network Architecture (SpiNNaker) provided by the University of Manchester. 

\subsection{SpiNNaker}
[TODO: Matt]
The SpiNNaker is cool. 
Here's why. (Energy efficient)
It's so epic. 
And we have one. 
And we added a cool functionality to it. 

\subsection{Synaptogensis}
[TODO: Matt]
Define Synaptogenesis and explain why it's cool.
Here's why. 
Explain it's previous success from Levy's work.

\hfill

[TODO: Matt] We're combining the two. And it's gonna be super useful.


\section{Design}
\IEEEPARstart{N}{eural} networks consist of biologically inspired relationships between individual nodes (neurons). Learning in such a network occurs by intelligently adjusting weights on the links between the nodes. We define the relationship between nodes into the following three categories:

\begin{itemize}
\item[(a)] {\bf Static} - where the network topology is fixed, and can only change by manually adjusting the relationship between nodes. Most neural networks fall into this category because regardless of the number of nodes \(n\), only one fixed relationship forms between them:

\[
\text{NumberOfTopologies}(n) = 1
\]


\item[(b)] {\bf Semi-Dynamic} - in which the network is not static, and there is some freedom for nodes to rewire with other nodes in real time. The complexity grows exponentially. Each node has non-zero probability to be rewired with a constant \(c\) number of other nodes. Given \(n\) such nodes, the number of possible topologies can be up to

\[
\text{NumberOfTopologies}(n) = c^n
\]


\item[(c)] {\bf Dynamic} - where each node has non-zero probability to be wired with any other neuron in the entire network. Given \(n\) nodes, the number of possible topologies becomes

\[
\text{NumberOfTopologies}(n) = n^{n-1}
\]


\end{itemize}

A biological neural network such as the human brain does not follow the static network topology. Connections between neurons are regularly formed and removed over time. Moreover, such a neural network is not fully dynamic either, since locality is a physical constraint. For example, a neuron on the far end of the left hemisphere of a brain may never directly connect with a neuron on the right hemisphere.

We designed a programming framework on SpiNNaker's API to enable a semi-dynamic network topology. In our implementation, each neuron had the flexibility to rewire with none, all, or some of the fixed number of other neurons. 

\section{Implementation}
\IEEEPARstart{B}{y} taking advantage of the multiple independent cores on the SpiNNaker, we were able to emulate efficient semi-dynamic synaptogenesis. We implemented a perceptron with fifteen nodes consisting of nine input layers and six output layers. As diagrammed in Figure 1 below, each of the nine neurons broadcasted its excitation to an arbitrarily chosen six output neurons. 

\vspace{0.5cm}
[Figure 1]
\vspace{0.5cm}

We used the weight modification rule proposed by Levy to categorize an input set of letters. 

\vspace{0.5cm}
[Delta W rule]
\vspace{0.5cm}

We implemented a single-layer perception as a proof of concept to demonstrate synaptogenesis on the SpiNNaker board. In our network, every input node broadcasts a data packet to six of the output nodes. To simulate the creation and rewiring of synapses, not all messages are registered by the output neurons. 

Each packet received by an output node is looked up in the output node's local list of incoming connections. If the address from the packet is not listed in the incoming connections list, it will be ignored, and no further computation will be done.

Synaptic connectivity is broken when weights fall below some negative threshold. More specifically, when an average running weight dropped below \(-\delta\) for some \(\delta > 0\), the link is considered disconnected.

New connections are established between nodes when an average rate of activity drops below 25\%. In this case, the next data packet received from a muted neuron becomes unmuted, in hopes to raise the average rate of activity.

\section{Pitfalls}
\IEEEPARstart{S}{ome} disadvantages are present when considering our approach for synaptogenesis on the SpiNNaker. [TODO: Matt] Blah blah.


\section{Further Study}
\IEEEPARstart{T}{his} paper reveals multiple questions that still need to be examined further. 


\section{Conclusion}
The conclusion goes here.


% if have a single appendix:
%\appendix[Proof of the Zonklar Equations]
% or
%\appendix  % for no appendix heading
% do not use \section anymore after \appendix, only \section*
% is possibly needed

% use appendices with more than one appendix
% then use \section to start each appendix
% you must declare a \section before using any
% \subsection or using \label (\appendices by itself
% starts a section numbered zero.)
%


\appendices
\section{Shannon's Entropy}
Appendix one text goes here.

\section{Calculation of Mutual Information}
Appendix one text goes here.

% you can choose not to have a title for an appendix
% if you want by leaving the argument blank
\section{Calculation of Statistical Dependence}
Appendix two text goes here.


% use section* for acknowledgement
\section*{Acknowledgment}


The authors would like to thank Professor Worthy Martin, Associate Professor of Computer Science at the University of Virginia.


% Can use something like this to put references on a page
% by themselves when using endfloat and the captionsoff option.
\ifCLASSOPTIONcaptionsoff
  \newpage
\fi



% trigger a \newpage just before the given reference
% number - used to balance the columns on the last page
% adjust value as needed - may need to be readjusted if
% the document is modified later
%\IEEEtriggeratref{8}
% The "triggered" command can be changed if desired:
%\IEEEtriggercmd{\enlargethispage{-5in}}

% references section

% can use a bibliography generated by BibTeX as a .bbl file
% BibTeX documentation can be easily obtained at:
% http://www.ctan.org/tex-archive/biblio/bibtex/contrib/doc/
% The IEEEtran BibTeX style support page is at:
% http://www.michaelshell.org/tex/ieeetran/bibtex/
%\bibliographystyle{IEEEtran}
% argument is your BibTeX string definitions and bibliography database(s)
%\bibliography{IEEEabrv,../bib/paper}
%
% <OR> manually copy in the resultant .bbl file
% set second argument of \begin to the number of references
% (used to reserve space for the reference number labels box)
\begin{thebibliography}{1}

\bibitem{IEEEhowto:kopka}
H.~Kopka and P.~W. Daly, \emph{A Guide to \LaTeX}, 3rd~ed.\hskip 1em plus
  0.5em minus 0.4em\relax Harlow, England: Addison-Wesley, 1999.

\end{thebibliography}

\end{document}


